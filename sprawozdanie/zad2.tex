\section{Zadanie 2}
\subsection{Treść} 
Wygeneruj losowy obraz binarny o rozmiarze n$\times$n i wybranej gęstości $\delta$ czarnych punktów
$\delta$ = 0.1, 0.3, 0.4. Korzystając z różnego typu sąsiedztwa (4-sąsiadów, 8-sąsiadów,
8-16-sąsiadów) zaproponuj funkcję energii (np. w bliskiej odległości te same kolory przyciągają
się, a w dalszej odpychają się) i dokonaj jej minimalizacji za pomocą algorytmu
symulowanego wyżarzania. W jaki sposób można generować stany sąsiednie? Jak róż-
nią się uzyskane wyniki w zależności od rodzaju sąsiedztwa, wybranej funkcji energii i
szybkości spadku temperatury?
\section{Kod}
Kod odpowiedzialny za problem krystalizacji znajduje się w TODO\_repo.
\begin{itemize}
\item Najpierw generowany jest losowy obraz zawierający określoną ilość punktów.
\item Następny stan jest generowany przez zamianę losowych dwóch punktów ze sobą, niezależnie od ich koloru.
\item Energia jest obliczana jako suma energii każdego czarnego punktu na podstawie typu sąsiedztwa.
\item Warunkiem zatrzymania jest jedynie liczba iteracji; przy implementacji uznałem że minimalna energia jest nieznana i nie sądziłem, że energia może osiągnąć 0 jak to miało miejsce w kilku przypadkach.
\end{itemize}

\subsection{Wybór funkcji temperatury}
Najpierw, podobnie jak w poprzednim zadaniu, wybrałem funkcję temperatury.
Ustaliłem następujące parametry:
\begin{itemize}
\item Temperatura początkowa: 10000
\item Liczba iteracji: 500000
\item Rozmiar obrazu: 256
\item Gęstość: 0.3
\item Funkcja temperatury: ?
\item Typ sąsiedztwa: krzyż 
\end{itemize}

\begin{minipage}{\textwidth}
\begin{minipage}{.7\linewidth}
$T(i) = T_0\cdot(1-0.00001)^{i}$\\
  \includegraphics[width=\textwidth]{../images/crystal_1sf_03_10000_Cross_exp00001_500000_plot.png}
\end{minipage}%
\begin{minipage}{.3\linewidth}
  \includegraphics[width=\textwidth]{crystal_sf/crystal_1sf_03_10000_Cross_exp00001_500000_initial.png}
  \includegraphics[width=\textwidth]{crystal_sf/crystal_1sf_03_10000_Cross_exp00001_500000_result.png}
\end{minipage}
\end{minipage}

\begin{minipage}{\textwidth}
$T(i) = T_0\cdot(1-0.0001)^{i}$\\
\begin{minipage}{.7\linewidth}
  \includegraphics[width=\textwidth]{crystal_sf/crystal_1sf_03_10000_Cross_exp0001_500000_plot.png}
\end{minipage}%
\begin{minipage}{.3\linewidth}
  \includegraphics[width=\textwidth]{crystal_sf/crystal_1sf_03_10000_Cross_exp0001_500000_initial.png}
  \includegraphics[width=\textwidth]{crystal_sf/crystal_1sf_03_10000_Cross_exp0001_500000_result.png}
\end{minipage}
\end{minipage}

\begin{minipage}{\textwidth}
$T(i) = T_0\cdot(1-0.001)^{i}$\\
\begin{minipage}{.7\linewidth}
  \includegraphics[width=\textwidth]{crystal_sf/crystal_1sf_03_10000_Cross_exp001_500000_plot.png}
\end{minipage}%
\begin{minipage}{.3\linewidth}
  \includegraphics[width=\textwidth]{crystal_sf/crystal_1sf_03_10000_Cross_exp001_500000_initial.png}
  \includegraphics[width=\textwidth]{crystal_sf/crystal_1sf_03_10000_Cross_exp001_500000_result.png}
\end{minipage}
\end{minipage}

\begin{minipage}{\textwidth}
$T(i) = T_0\cdot(1-0.01)^{i}$\\
\begin{minipage}{.7\linewidth}
  \includegraphics[width=\textwidth]{crystal_sf/crystal_1sf_03_10000_Cross_exp01_500000_plot.png}
\end{minipage}%
\begin{minipage}{.3\linewidth}
  \includegraphics[width=\textwidth]{crystal_sf/crystal_1sf_03_10000_Cross_exp01_500000_initial.png}
  \includegraphics[width=\textwidth]{crystal_sf/crystal_1sf_03_10000_Cross_exp01_500000_result.png}
\end{minipage}
\end{minipage}

\begin{minipage}{\textwidth}
$T(i) = \frac{-i\cdot T_0}{i_{max}} + T_0$\\
\begin{minipage}{.7\linewidth}
  \includegraphics[width=\textwidth]{crystal_sf/crystal_1sf_03_10000_Cross_linear_500000_plot.png}
\end{minipage}%
\begin{minipage}{.3\linewidth}
  \includegraphics[width=\textwidth]{crystal_sf/crystal_1sf_03_10000_Cross_linear_500000_initial.png}
  \includegraphics[width=\textwidth]{crystal_sf/crystal_1sf_03_10000_Cross_linear_500000_result.png}
\end{minipage}
\end{minipage}

Pomimo wyglądu wykresu energii do dalszych testów wybrałem pierwszą funkcję, czyli
$T(i) = T_0\cdot(1-0.0001)^{i}$.
\subsection{Wybór energii początkowej}
Po wybraniu funkcji sprawdziłem jaką początkową temperaturę najlepiej wybrać do symulacji.
Parametry:
\begin{itemize}
\item Temperatura początkowa: ?
\item Liczba iteracji: 500000
\item Rozmiar obrazu: 256
\item Gęstość: 0.3
\item Funkcja temperatury: $T(i) = T_0(1-0.00001)^{i}$
\item Typ sąsiedztwa: krzyż 
\end{itemize}

$T_0=10000$\\
\begin{minipage}{\textwidth}
\begin{minipage}{.7\linewidth}
  \includegraphics[width=\textwidth]{crystal_sf/crystal_1sf_03_10000_Cross_exp00001_500000_plot.png}
\end{minipage}%
\begin{minipage}{.3\linewidth}
  \includegraphics[width=\textwidth]{crystal_sf/crystal_1sf_03_10000_Cross_exp00001_500000_initial.png}
  \includegraphics[width=\textwidth]{crystal_sf/crystal_1sf_03_10000_Cross_exp00001_500000_result.png}
\end{minipage}
\end{minipage}

\begin{minipage}{\textwidth}
\begin{minipage}{.7\linewidth}
$T_0=1000$\\
  \includegraphics[width=\textwidth]{crystal_sf/crystal_1sf_03_1000_Cross_exp00001_500000_plot.png}
\end{minipage}%
\begin{minipage}{.3\linewidth}
  \includegraphics[width=\textwidth]{crystal_sf/crystal_1sf_03_1000_Cross_exp00001_500000_initial.png}
  \includegraphics[width=\textwidth]{crystal_sf/crystal_1sf_03_1000_Cross_exp00001_500000_result.png}
\end{minipage}
\end{minipage}

\begin{minipage}{\textwidth}
\begin{minipage}{.7\linewidth}
$T_0=100$\\
  \includegraphics[width=\textwidth]{crystal_sf/crystal_1sf_03_100_Cross_exp00001_500000_plot.png}
\end{minipage}%
\begin{minipage}{.3\linewidth}
  \includegraphics[width=\textwidth]{crystal_sf/crystal_1sf_03_100_Cross_exp00001_500000_initial.png}
  \includegraphics[width=\textwidth]{crystal_sf/crystal_1sf_03_100_Cross_exp00001_500000_result.png}
\end{minipage}
\end{minipage}

\begin{minipage}{\textwidth}
\begin{minipage}{.7\linewidth}
$T_0=10$\\
  \includegraphics[width=\textwidth]{crystal_sf/crystal_1sf_03_10_Cross_exp00001_500000_plot.png}
\end{minipage}%
\begin{minipage}{.3\linewidth}
  \includegraphics[width=\textwidth]{crystal_sf/crystal_1sf_03_10_Cross_exp00001_500000_initial.png}
  \includegraphics[width=\textwidth]{crystal_sf/crystal_1sf_03_10_Cross_exp00001_500000_result.png}
\end{minipage}
\end{minipage}

\begin{minipage}{\textwidth}
\begin{minipage}{.7\linewidth}
$T_0=1$\\
  \includegraphics[width=\textwidth]{crystal_sf/crystal_1sf_03_1_Cross_exp00001_500000_plot.png}
\end{minipage}%
\begin{minipage}{.3\linewidth}
  \includegraphics[width=\textwidth]{crystal_sf/crystal_1sf_03_1_Cross_exp00001_500000_initial.png}
  \includegraphics[width=\textwidth]{crystal_sf/crystal_1sf_03_1_Cross_exp00001_500000_result.png}
\end{minipage}
\end{minipage}

\begin{minipage}{\textwidth}
\begin{minipage}{.7\linewidth}
$T_0=0.5$\\
  \includegraphics[width=\textwidth]{crystal_sf/crystal_1sf_03_0_5_Cross_exp00001_500000_plot.png}
\end{minipage}%
\begin{minipage}{.3\linewidth}
  \includegraphics[width=\textwidth]{crystal_sf/crystal_1sf_03_0_5_Cross_exp00001_500000_initial.png}
  \includegraphics[width=\textwidth]{crystal_sf/crystal_1sf_03_0_5_Cross_exp00001_500000_result.png}
\end{minipage}
\end{minipage}

\begin{minipage}{\textwidth}
\begin{minipage}{.7\linewidth}
$T_0=0.1$\\
  \includegraphics[width=\textwidth]{crystal_sf/crystal_1sf_03_0_1_Cross_exp00001_500000_plot.png}
\end{minipage}%
\begin{minipage}{.3\linewidth}
  \includegraphics[width=\textwidth]{crystal_sf/crystal_1sf_03_0_1_Cross_exp00001_500000_initial.png}
  \includegraphics[width=\textwidth]{crystal_sf/crystal_1sf_03_0_1_Cross_exp00001_500000_result.png}
\end{minipage}
\end{minipage}
Spośród powyższych energii najlepszą charakterystykę wydaje się dawać energia początkowa równa 100, jednak liczba iteracji jest zbyt mała.