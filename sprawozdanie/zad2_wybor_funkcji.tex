\subsection{Wybór funkcji temperatury}
Najpierw, podobnie jak w poprzednim zadaniu, wybrałem funkcję temperatury.
Ustaliłem następujące parametry:
\begin{itemize}
\item Temperatura początkowa: 10000
\item Liczba iteracji: 500000
\item Rozmiar obrazu: 256
\item Gęstość: 0.3
\item Funkcja temperatury: ?
\item Typ sąsiedztwa: krzyż 
\end{itemize}

\begin{minipage}{\textwidth}
\begin{minipage}{.7\linewidth}
$T(i) = T_0\cdot(1-0.00001)^{i}$\\
  \includegraphics[width=\textwidth]{../images/crystal_1sf_03_10000_Cross_exp00001_500000_plot.png}
\end{minipage}%
\begin{minipage}{.3\linewidth}
  \includegraphics[width=\textwidth]{crystal_sf/crystal_1sf_03_10000_Cross_exp00001_500000_initial.png}
  \includegraphics[width=\textwidth]{crystal_sf/crystal_1sf_03_10000_Cross_exp00001_500000_result.png}
\end{minipage}
\end{minipage}

\begin{minipage}{\textwidth}
$T(i) = T_0\cdot(1-0.0001)^{i}$\\
\begin{minipage}{.7\linewidth}
  \includegraphics[width=\textwidth]{crystal_sf/crystal_1sf_03_10000_Cross_exp0001_500000_plot.png}
\end{minipage}%
\begin{minipage}{.3\linewidth}
  \includegraphics[width=\textwidth]{crystal_sf/crystal_1sf_03_10000_Cross_exp0001_500000_initial.png}
  \includegraphics[width=\textwidth]{crystal_sf/crystal_1sf_03_10000_Cross_exp0001_500000_result.png}
\end{minipage}
\end{minipage}

\begin{minipage}{\textwidth}
$T(i) = T_0\cdot(1-0.001)^{i}$\\
\begin{minipage}{.7\linewidth}
  \includegraphics[width=\textwidth]{crystal_sf/crystal_1sf_03_10000_Cross_exp001_500000_plot.png}
\end{minipage}%
\begin{minipage}{.3\linewidth}
  \includegraphics[width=\textwidth]{crystal_sf/crystal_1sf_03_10000_Cross_exp001_500000_initial.png}
  \includegraphics[width=\textwidth]{crystal_sf/crystal_1sf_03_10000_Cross_exp001_500000_result.png}
\end{minipage}
\end{minipage}

\begin{minipage}{\textwidth}
$T(i) = T_0\cdot(1-0.01)^{i}$\\
\begin{minipage}{.7\linewidth}
  \includegraphics[width=\textwidth]{crystal_sf/crystal_1sf_03_10000_Cross_exp01_500000_plot.png}
\end{minipage}%
\begin{minipage}{.3\linewidth}
  \includegraphics[width=\textwidth]{crystal_sf/crystal_1sf_03_10000_Cross_exp01_500000_initial.png}
  \includegraphics[width=\textwidth]{crystal_sf/crystal_1sf_03_10000_Cross_exp01_500000_result.png}
\end{minipage}
\end{minipage}

\begin{minipage}{\textwidth}
$T(i) = \frac{-i\cdot T_0}{i_{max}} + T_0$\\
\begin{minipage}{.7\linewidth}
  \includegraphics[width=\textwidth]{crystal_sf/crystal_1sf_03_10000_Cross_linear_500000_plot.png}
\end{minipage}%
\begin{minipage}{.3\linewidth}
  \includegraphics[width=\textwidth]{crystal_sf/crystal_1sf_03_10000_Cross_linear_500000_initial.png}
  \includegraphics[width=\textwidth]{crystal_sf/crystal_1sf_03_10000_Cross_linear_500000_result.png}
\end{minipage}
\end{minipage}

Pomimo wyglądu wykresu energii do dalszych testów wybrałem pierwszą funkcję, czyli
$T(i) = T_0\cdot(1-0.0001)^{i}$.