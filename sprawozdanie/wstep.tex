\section{Wstęp}
Kod programu znajduje się w repozytorium \url{https://github.com/grpawel/mownit-lab4-annealing}.
Algorytm zaimplementowany jest w klasie \href{https://github.com/grpawel/mownit-lab4-annealing/blob/master/src/pl/edu/agh/mownit/lab4/annealing/AnnealingSimulator.java}{AnnealingSimulator}. Ustawienia symulacji pobierane są z pliku w \href{https://github.com/grpawel/mownit-lab4-annealing/blob/master/src/pl/edu/agh/mownit/lab4/annealing/AnnealingSettingsReader.java}{AnnealingSettingsReader}. Poszczególne problemy i ich implementacje (oblicznanie energii, wybór stanu następnego) są w odpowiednich pakietach.
Algorytm ma dwa warunki stopu - liczbę iteracji oraz osiągnięcie rezultatu (dla sudoku). Nie sprawdzam minimalnej temperatury, ponieważ część funkcji temperatury dopasowuje się do maksymalnej liczby iteracji.