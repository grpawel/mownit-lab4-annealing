\section{Zadanie 1}
\subsection{Treść}
Wygeneruj chmurę n losowych punktów w 2D, a następnie zastosuj algorytm symulowanego
wyżarzania do przybliżonego rozwiązania problemu komiwojażera dla tych punktów.\\
a) Przedstaw wizualizację otrzymanego rozwiązania dla 3 różnych wartości n oraz 3
różnych układów punktów w 2D (rozkład jednostajny, rozkład normalny z czterema
różnymi grupami parametrów, dziewięć odseparowanych grup punktów).\\
b) Zbadaj wpływ sposobu generacji sąsiedniego stanu (consecutive swap vs. arbitrary
swap) oraz funkcji zmiany temperatury na zbieżność procesu optymalizacji.\\
c) Przedstaw wizualizację (saoptimset) działania procedury minimalizującej funkcję
celu.
\subsection{Wstęp}
Niestety w tym zadaniu miałem problem ze zbliżaniem się do energii minimalnej dla większej ilości punktów. W pewnym momencie żaden następny stan nie osiąga niższej energii niż poprzedni. Nie byłem w stanie znaleźć źródła problemu. Problem występuje niezależnie od parametrów symulacji. Jedynie dla niewielkiej ilości punktów znajdywane jest optymalne rozwiązanie.

Miasta przechowywane są w liście w kolejności odwiedzania. Przy generowaniu stanu następnego zamieniam dwa elementy w liście ze sobą, albo sąsiednie, albo losowe. Funkcja energii to suma odległości kolejnych par punktów.
\subsection{Rozkład normalny}
Parametry:
\begin{itemize}
    \item Temperatura początkowa: 100
    \item Liczba iteracji: 1000000
    \item Rozmiar obrazu: 512x512
    \item Funkcja temperatury: $T(i) = \frac{-i\cdot T_0}{i_{max}} + T_0$ (f. liniowa)
    \item Liczba punktów: ?
    \item Rozkład punktów: jednostajny
    \item Generowanie sąsiedniego stanu: consecutive
\end{itemize}
\bigresult{tsp_512_10_cons_unif_exp00001_100_1000000}{Energia początkowa: 2529.3, końcowa: 1403.7, liczba punktów: 10\\}
\bigresult{tsp_512_100_cons_unif_exp00001_100_1000000}{Energia początkowa: 25384.3, końcowa: 21638.8, liczba punktów: 100\\}
\bigresult{tsp_512_1000_cons_unif_exp00001_100_1000000}{Energia początkowa: 262692.9, końcowa: 244416.6, liczba punktów: 1000\\}


\subsection{Odseparowane grupy}
Parametry:
\begin{itemize}
    \item Temperatura początkowa: 100
    \item Liczba iteracji: 1000000
    \item Rozmiar obrazu: 512x512
    \item Funkcja temperatury:$T(i) = T_0\cdot(1-0.00001)^{i}$
    \item Liczba punktów: ?
    \item Rozkład punktów: jednostajny
    \item Generowanie sąsiedniego stanu: consecutive
\end{itemize}
\bigresult{tsp_512_10_cons_mugr_exp00001_100_1000000}{Energia początkowa: 3365.0, końcowa: 1901.2, liczba punktów: 10}
\bigresult{tsp_512_100_cons_mugr_exp00001_100_1000000}{Energia początkowa: 29691.7, końcowa: 24443.6, liczba punktów: 100}
\bigresult{tsp_512_1000_cons_mugr_exp00001_100_1000000}{Energia początkowa: 296512.5, końcowa: 282814.1, liczba punktów: 1000\\}


\subsection{Sposób generowania sąsiedniego stanu}
Parametry:
\begin{itemize}
    \item Temperatura początkowa: 100
    \item Liczba iteracji: 1000000
    \item Rozmiar obrazu: 512x512
    \item Funkcja temperatury: $T(i) = T_0\cdot(1-0.00001)^{i}$
    \item Liczba punktów: ?
    \item Rozkład punktów: jednostajny
    \item Generowanie sąsiedniego stanu: ?
\end{itemize}
\bigresult{tsp_512_10_cons_unif_exp00001_100_1000000_a}{consecutive, liczba punktów: 10, energia początkowa: 3028.9, końcowa: 1438.3\\}
\bigresult{tsp_512_10_arbi_unif_exp00001_100_1000000_a}{arbitrary, liczba punktów: 10, energia początkowa: 3028.9, końcowa: 1438.3\\}
\bigresult{tsp_512_100_cons_unif_exp00001_100_1000000_a}{consecutive, liczba punktów: 10, energia początkowa: 26274.7, końcowa: 21225.0\\}
\bigresult{tsp_512_100_arbi_unif_exp00001_100_1000000_a}{arbitrary, liczba punktów: 10, energia początkowa: 26274.7, końcowa: 21093.8\\}
\bigresult{tsp_512_1000_cons_unif_exp00001_100_1000000_a}{consecutive, liczba punktów: 10, energia początkowa: 264528.4, końcowa: 248557.4\\}
\bigresult{tsp_512_1000_arbi_unif_exp00001_100_1000000_a}{arbitrary, liczba punktów: 10, energia początkowa: 264528.4, końcowa: 247855.2\\}
Niestety nie widać większej różnicy z powodu niezbiegania do minimum funkcji energii. W poprawnie działającym programie strategia consecutive powinna dawać lepsze rezultaty.

\subsection{Różne funkcje temperatury}
Parametry:
\begin{itemize}
    \item Temperatura początkowa: 100
    \item Liczba iteracji: 1000000
    \item Rozmiar obrazu: 512x512
    \item Gęstość: 0.3
    \item Funkcja temperatury: ?
    \item Liczba punktów: 10
    \item Generowanie sąsiedniego stanu: arbitrary
\end{itemize}
\bigplot{tsp_512_10_arbi_unif_linear_100_1000000_b}{$T(i) = \frac{-i\cdot T_0}{i_{max}} + T_0$(f. liniowa), energia początkowa: 3815.8, końcowa: 1689.6\\}
\bigplot{tsp_512_10_arbi_unif_exp00001_100_1000000_b}{$T(i) = T_0\cdot(1-0.00001)^{i}$, energia początkowa: 1989.2, końcowa: 1220.9\\}
\bigplot{tsp_512_10_arbi_unif_exp0001_100_1000000_b}{$T(i) = T_0\cdot(1-0.0001)^{i}$, energia początkowa: 2329.5, końcowa: 1533.8\\}
\bigplot{tsp_512_10_arbi_unif_exp001_100_1000000_b}{$T(i) = T_0\cdot(1-0.001)^{i}$, energia początkowa: 2553.6, końcowa: 1602.9\\}
\bigplot{tsp_512_10_arbi_unif_exp01_100_1000000_b}{$T(i) = T_0\cdot(1-0.01)^{i}$, energia początkowa: 2974.1, końcowa: 1487.6\\}
\bigplot{tsp_512_10_arbi_unif_constant_100_1000000_b}{$T(i) = T_0$, energia początkowa: 2855.1, końcowa: 1517.3\\}
Im temperatura szybciej maleje, tym szybciej mniej optymalne stany przestają być akceptowane
