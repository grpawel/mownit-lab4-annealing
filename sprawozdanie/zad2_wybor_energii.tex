\subsection{Wybór energii początkowej}
Po wybraniu funkcji sprawdziłem jaką początkową temperaturę najlepiej wybrać do symulacji.
Parametry:
\begin{itemize}
\item Temperatura początkowa: ?
\item Liczba iteracji: 500000
\item Rozmiar obrazu: 256
\item Gęstość: 0.3
\item Funkcja temperatury: $T(i) = T_0(1-0.00001)^{i}$
\item Typ sąsiedztwa: krzyż 
\end{itemize}

$T_0=10000$\\
\begin{minipage}{\textwidth}
\begin{minipage}{.7\linewidth}
  \includegraphics[width=\textwidth]{crystal_sf/crystal_1sf_03_10000_Cross_exp00001_500000_plot.png}
\end{minipage}%
\begin{minipage}{.3\linewidth}
  \includegraphics[width=\textwidth]{crystal_sf/crystal_1sf_03_10000_Cross_exp00001_500000_initial.png}
  \includegraphics[width=\textwidth]{crystal_sf/crystal_1sf_03_10000_Cross_exp00001_500000_result.png}
\end{minipage}
\end{minipage}

\begin{minipage}{\textwidth}
\begin{minipage}{.7\linewidth}
$T_0=1000$\\
  \includegraphics[width=\textwidth]{crystal_sf/crystal_1sf_03_1000_Cross_exp00001_500000_plot.png}
\end{minipage}%
\begin{minipage}{.3\linewidth}
  \includegraphics[width=\textwidth]{crystal_sf/crystal_1sf_03_1000_Cross_exp00001_500000_initial.png}
  \includegraphics[width=\textwidth]{crystal_sf/crystal_1sf_03_1000_Cross_exp00001_500000_result.png}
\end{minipage}
\end{minipage}

\begin{minipage}{\textwidth}
\begin{minipage}{.7\linewidth}
$T_0=100$\\
  \includegraphics[width=\textwidth]{crystal_sf/crystal_1sf_03_100_Cross_exp00001_500000_plot.png}
\end{minipage}%
\begin{minipage}{.3\linewidth}
  \includegraphics[width=\textwidth]{crystal_sf/crystal_1sf_03_100_Cross_exp00001_500000_initial.png}
  \includegraphics[width=\textwidth]{crystal_sf/crystal_1sf_03_100_Cross_exp00001_500000_result.png}
\end{minipage}
\end{minipage}

\begin{minipage}{\textwidth}
\begin{minipage}{.7\linewidth}
$T_0=10$\\
  \includegraphics[width=\textwidth]{crystal_sf/crystal_1sf_03_10_Cross_exp00001_500000_plot.png}
\end{minipage}%
\begin{minipage}{.3\linewidth}
  \includegraphics[width=\textwidth]{crystal_sf/crystal_1sf_03_10_Cross_exp00001_500000_initial.png}
  \includegraphics[width=\textwidth]{crystal_sf/crystal_1sf_03_10_Cross_exp00001_500000_result.png}
\end{minipage}
\end{minipage}

\begin{minipage}{\textwidth}
\begin{minipage}{.7\linewidth}
$T_0=1$\\
  \includegraphics[width=\textwidth]{crystal_sf/crystal_1sf_03_1_Cross_exp00001_500000_plot.png}
\end{minipage}%
\begin{minipage}{.3\linewidth}
  \includegraphics[width=\textwidth]{crystal_sf/crystal_1sf_03_1_Cross_exp00001_500000_initial.png}
  \includegraphics[width=\textwidth]{crystal_sf/crystal_1sf_03_1_Cross_exp00001_500000_result.png}
\end{minipage}
\end{minipage}

\begin{minipage}{\textwidth}
\begin{minipage}{.7\linewidth}
$T_0=0.5$\\
  \includegraphics[width=\textwidth]{crystal_sf/crystal_1sf_03_0_5_Cross_exp00001_500000_plot.png}
\end{minipage}%
\begin{minipage}{.3\linewidth}
  \includegraphics[width=\textwidth]{crystal_sf/crystal_1sf_03_0_5_Cross_exp00001_500000_initial.png}
  \includegraphics[width=\textwidth]{crystal_sf/crystal_1sf_03_0_5_Cross_exp00001_500000_result.png}
\end{minipage}
\end{minipage}

\begin{minipage}{\textwidth}
\begin{minipage}{.7\linewidth}
$T_0=0.1$\\
  \includegraphics[width=\textwidth]{crystal_sf/crystal_1sf_03_0_1_Cross_exp00001_500000_plot.png}
\end{minipage}%
\begin{minipage}{.3\linewidth}
  \includegraphics[width=\textwidth]{crystal_sf/crystal_1sf_03_0_1_Cross_exp00001_500000_initial.png}
  \includegraphics[width=\textwidth]{crystal_sf/crystal_1sf_03_0_1_Cross_exp00001_500000_result.png}
\end{minipage}
\end{minipage}
Spośród powyższych energii najlepszą charakterystykę wydaje się dawać energia początkowa równa 100, jednak liczba iteracji jest zbyt mała.